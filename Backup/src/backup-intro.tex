Onze wereld draait inmiddels om de data die we genereren. Het zijn de documenten op ons werk, de berichten op social media, de foto's op onze telefoons. Ons hele leven hangt van digitale data aan elkaar.

Onze data staat op apparaten en daar kan van alles mee gebeuren:
\begin{description}
\item [menselijke fouten] Het per ongeluk weggegooien van data, of later denken dat het toch niet weg had gemoeten
\item [natuurlijke oorzaken] Door brand of wateroverlast kunnen opslagsystemen verloren gaan of onherstelbaar beschadigd worden
\item [ouderdom] - Een harddisk of SSD kan stuk gaan door ouderdom
\item [bitrot] - Data wordt opgeslagen op een harddisk met magnetisme dat verliest na verloop van tijd zijn magnetisme en op een gegeven moment is er geen veschil meer tussen een 1 en een 0
\item [diefstal] - Apparatuur waarop we belangrijke data hebben staan kunnen gestolen worden
\item [cryptolockers] - Malware kan onze data encrypten en alleen tegen betaling krijgen we misschien de sleutel om de data te decrypten
\end{description}

Het is steeds de vraag waartegen we ons willen beschermen, wat het mag kosten en wat de eisen zijn die we stellen aan onze data. Om een goede afweging te maken is het belangrijk dat we de verschillende mogelijkheden kennen.

