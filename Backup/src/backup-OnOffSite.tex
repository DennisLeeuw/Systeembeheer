Met een kopie van de data op dezelfde harddisk heeft dit gevolgen als de harddisk overlijdt. Het is beter om een kopie te maken op een andere harddisk. Zolang als de data binnen dezelfde lokatie blijft heet dit een on-site backup. Want als de lokatie afbrandt hebben we geen kopie meer, die is dan vermoedelijk ook verbrand.

Een oplossing hiervoor is het overbrengen van de kopie naar een andere lokatie. Dit heet dan een off-site backup. De vraag reist nu hoe ver weg moet de backup zijn om veilig te zijn? Om deze vraag te kunnen beantwoorden moeten we ons afvragen waartegen we ons willen beschermen. Is het voldoende om ons te beschermen tegen een brand op een lokatie of moeten we de data beschermen tegen een nucleaire aanval? De keuze is vaak afhankelijk van de waarde van de data, de gevolgen bij het verloren gaan en de kosten van transport en opslag.

