Een kopie van de data kan bijvoorbeeld met \texttt{rsync} gemaakt worden van \'e\'en systeem naar een tweede. Om dit te kunnen doen moeten beide systemen aan staan. Ook bij RAID kunnen we data repliceren (mirror) naar een tweede disk. In beide gevallen spreken we van on-line data. Het voordeel van deze manier is dat de data vrijwel meteen beschikbaar is als het originele systeem stuk gaat. Het nadeel is, omdat de data on-line is, dat bijvoorbeeld cryptoware of een andere aanvaller direct ook bij de data zou kunnen. Voor RAID is dit vooral een probleem omdat de data direct gerepliceerd wordt, dus de encrypte data wordt gerepliceerd. Een systeem dat periodiek een kopie maakt is hiervoor minder gevoelig als de plek waarnaar gekopieerd wordt niet van buitenaf benaderd kan worden.

Een andere mogelijkheid is dat data off-line wordt opgeslagen. Bij dit soort systemen denken we vaak aan tapes. De backup wordt gemaakt op tape en zodra de backup klaar is wordt de tape uit de tapestreamer verwijderd. De data is dan niet meer direct benaderbaar.

