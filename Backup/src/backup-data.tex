De meest eenvoudig vorm van backup is de kopie. Sla een kopie van een document op onder een andere naam of in een andere directorie en je kan deze terug zetten als er wat fout gegaan is met het origineel. Het nadeel van deze oplossing is dat als het opslagmedium in het systeem stuk gaat je het origineel en de kopie kwijt bent.

Een andere oplossing is het opslaan van een kopie van de data op een andere harddisk of op een USB-stick. Mocht het origineel verloren gaan of zelfs de volledige harddisk overlijden, dan heb je altijd nog een backup op de andere disk. Het is natuurlijk wel arbeidsintensief om elke keer een kopie te maken als je het origineel hebt aangepast of als je een nieuw document hebt gemaakt. Het gebeurt natuurlijk een keer dat je dat vergeet.

We zouden ook een stukje software elk uur kunnen laten draaien om een backup te maken van al onze documenten. Zo weten we zeker dat er altijd een backup is van maximaal een uur oud. Het nadeel is dat als we terug willen naar de versie van 3 uur geleden, dat die inmiddels overschreven is door de versie van 2 uur geleden.

We willen dus een kopie hebben die afhankelijk is van het tijdstip waarop de backup gemaakt is. Dat kan door backup-software een map te laten maken met in de naam de datum en het tijdstip. Daarin kan deze een kopie zetten van onze data, dan kunnen we kiezen welke versie we later terug willen zetten. Het nadeel is dat we elk uur een nieuwe map krijgen met al onze data erin. De harddisk zal dus heel snel vol lopen.

Voor dit laatste probleem zijn er verschillende oplossingen:
\begin{enumerate}
\item Is het echt nodig om elk uur een kopie te maken of is \'e\'en keer per dag genoeg?
\item Limiteer het aantal te bewaren mappen. Hoe ver wil je terug kunnen in de tijd?
\item Maak alleen een kopie van die data die daadwerkelijk gewijzigd is
\item Comprimeer (zip) de data
\item Dedupliceer de data - veel gebruikers slaan dezelfde documenten op, dus als twee of meer documenten hetzelfde zijn sla er dan maar 1 op in de backup
\end{enumerate}

Al deze features zijn vaak onderdeel van de huidige backup-software oplossingen. Vele leveranciers en open source programmeurs hebben zich gestort op het schrijven van de beste backup oplossing. Het loont de moeite om de verschillende producten met elkaar te vergelijken voordat je een beslissing neemt welke backup software je wilt gaan gebruiken.

