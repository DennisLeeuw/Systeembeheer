Een van de oudste systemen om data op te slaan is tape. Tape is niets anders dan en plastic lint dat voorzien is van magnetisch materiaal waar door een schrijfkop enen of nullen op gezet worden. Een leeskop kan die enen en nullen weer terug lezen.

Het systeem bestaat uit tapes en een tapestreamer. De tapestreamer kan geladen worden met 1 of meer tapes. De tapestreamer schrijft of leest de tapes. Als een tape vol is kan deze vervangen worden door een lege waarop meer data geschreven kan worden. Als je data wil lezen van een oudere tape dan moet je dus eerst tapes wisselen om deze data te kunnen lezen.

Backup software kan een een kopie maken van je data en deze wegschrijven naar een tape. Als de software daarna de tapestreamer de opdracht geeft om de tape te ejecten (uit te werpen) dan is de data verder veilig tegen bijvoorbeeld overschrijven. Met een datum op de tape kunnen we altijd de data van die datum terug halen.

Tape kent een aantal voordelen:
\begin{itemize}
\item Relatief lage kosten van de tapes
\item Oneindig uitbreidbaar, we kunnen steeds nieuwe tapes kopen
\item Off-line, hacker en cryptolocker proof
\item Off-site bewaarbaar
\item Relatief lange levensduur
\item Energie zuinig, tapes hebben geen energie nodig om hun data te behouden
\end{itemize}

Natuurlijk zijn er ook nadelen aan tapes:
\begin{itemize}
\item Ze zijn relatief traag
\item Het kan wat werk zijn om de juiste tape terug te vinden
\item Tapes maken gebruik van magnetisme om hun data op te slaan, dus zijn ze gevoelig voor bitrot
\item Omdat ze data kunnen verliezen zullen tapes eens in de zoveel tijd geconroleert moeten worden of de data nog toegankelijk is
\item Het is een relatief arbeidsintensief proces, dus zijn de arbeidsloon kosten hoog
\item Wat vaak vergeten wordt zijn de kosten voor opslag. Tapes moeten ergens opgeborgen worden.
\end{itemize}

