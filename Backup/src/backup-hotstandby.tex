Voor hele servers is het lastig als er bij een calamiteit eerst een installatie van het operating system gedaan moet worden, daarna moet de benodigde software erop gezet worden en dan moet ook nog de data op het systeem gezet worden. Sommige servers zijn zo cruciaal voor een bedrijf dat dit te lang zou duren. Dit soort servers worden meestal voorzien van een broertje met een identieke installatie en de data wordt tussen de twee servers gesynchroniseerd. Omdat beide servers aan staan heet dit een Hot Standby. Zodra de productie server omvalt kan er meteen overgeschakeld worden op de Hot Standby die dan verder functioneert als de productie server. De defecte server zal dan vervangen moeten worden voor een nieuwe Hot Standby.

