RAT is een afkorting met een aantal mogelijke betekenissen, namelijk Remote Access Tool, Remote Administration Tool en Remote Access Trojan. Remote Access is een heel breed onderwerp, waar elke vorm van toegang tot systemen over een netwerk onder valt. In dit document zullen we ons beperken tot software waarmee we over een netwerk toegang tot een systeem kunnen krijgen om de configuratie van een machine aan te passen, Remote Administration dus. Remote Access Trojans worden behandeld in het \textquote{Security: Malware} document onder trojans.

Remote Administration is iets dat veel door systeembeheerders wordt gebruikt om servers in de serverruimte te configureren zonder dat er elke keer naar de serverruimte gelopen hoeft te worden. Er zijn ook software pakketten die het mogelijk maken om een compleet netwerk te voorzien van de juiste configuratie, dus ook voor het aansturen van routers en switches.

Remote Administration kan ook gebruikt worden door de cybercriminelen om machines remote te \textquote{beheren}. Deze malafide RAT software verbergt zich vaak op het systeem en is moeilijk terug te vinden. We kunnen deze software aantreffen op computers, maar ook op printers, smartphones of IoT-devices.

