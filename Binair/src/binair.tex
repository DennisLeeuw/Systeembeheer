Binair is een tweetalig stelsel, vandaar de bi in binair. We tellen dus met alleen de getallen 0 tot 1, dit in tegenstelling tot het decimale stelstel (deca is 10), waarbij we tellen van 0 tot 9.

Laten we bij het vertrouwde decimale stelsel beginnen. We tellen van 0 tot 9 en als er dan nog 1 bij komt (+1) dan zetten we een 1 voor de 0 (10) om aan te geven dat we eenkeer tot 9 hebben geteld. Het is dus tien. Tellen we nu weer door tot 9 (19) en tellen we er dan 1 bij op, dan wordt de voorste 1 een 2 (20) om aan te geven dat we voor de tweede keer tot 9 hebben geteld.

Dit zelfde principe kunnen we ook gebruiken voor binair, alleen tellen we van 0 tot 1 en daarna komt er een 1 voor, dan wordt het dus 10 en daarna 11 waarna we er weer een 1 voor moeten zetten, dus wordt het 100 en zo verder. De getallen 10 en 100 spreken we ook niet uit als tien en honderd, maar als \'e\'en nul en \'e\'en nul nul.

