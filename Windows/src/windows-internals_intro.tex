Of je nu programmeur, helpdesk medewerker, systeembeheerder, hacker of security officer bent het is handig om te weten hoe een systeem werkt. Als programmeur is het essentieel om te weten welke functies een systeem je aanbiedt zodat je er gebruik van kan maken. Als hacker is het goed om te weten wat het systeem je biedt en wat de vaak voorkomende fouten zijn die programmeurs maken. Voor helpdesk medewerkers is het handig als je meer weet van de werking van een systeem zodat je makkelijker de fouten van gebruikers snapt en voor systeembeheerders is de kennis van de interne werking van een operating systeem nodig zodat het systeem en de bijbehorende software op een juiste en veilige manier ge\"installeerd kan worden. Ten slotte is het voor de security officer essentieel om al deze voorkomende zaken te begrijpen, zodat een juiste security analyse gemaakt kan worden.

We beschrijven in dit stuk de globale werking van het Windows besturingssysteem.

