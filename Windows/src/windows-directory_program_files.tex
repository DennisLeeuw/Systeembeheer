De \texttt{Program Files}\index{Program Files} directory is de plek waar applicaties die niet onderdeel zijn van het besturingssysteem worden ge\"installeerd. Windows 64-bits kan ook Windows 32-bits applicaties draaien. Om deze twee soorten bestanden uit elkaar te houden zijn er op een 64-bits systeem twee \texttt{Program Files} directories, een 32-bits Windows systeem kent er maar \'e\'en.

De applicaties die gebouwd zijn met het aantal bits dat behoort bij het OS worden ge\"installeerd in de \texttt{\textbackslash Program Files}, voor een 32-bits systeem zijn dat de 32-bits applicaties en voor een 64-bits systeem zijn dat de 64-bits applicaties.

De 32-bits applicaties op een 64-bits systeem worden ge\"installeerd in de \texttt{\textbackslash Program Files (x86)} directory. Een 32-bits systeem kent deze directory niet.

