De \texttt{Windows} directory bevat alles wat bij het besturingssysteem behoort. Wijzigingen aanbrengen in deze tak van de directory tree kan je Windows systeem mogelijk stuk maken. Maar het is wel goed om te weten wat je zoal in deze tak kan aantreffen.

Windows gebruikt extensies om aan te geven om wat voor soort bestanden het gaat. De extentie .docx geeft aan dat het om en Microsoft Word bestand gaat in XML-formaat. Zo zijn er vele extensies, we zullen er hier een aantal noemen die je op het systeem regelmatig tegen zult komen:
\begin{description}
\item[exe] Een executable, dat wil zeggen dat het een commando of een applicatie is die je kan opstarten als je de juiste rechten daarvoor hebt.
\item[log] Een log bestand. Bevat informatie over wat er op het systeem gebeurd is.
\item[dll] Een dynamic loadable library. Wordt gebruikt door applicaties of commando's.
\item[txt] Een text bestand zonder opmaak van de tekst.
\item[ini] Een configuratie bestand. Wijzigingen in dit bestand kunnen de manier waarop het systeem werkt veranderen. Met het \texttt{type} commando kan je de inhoud van een ini of een txt bestand laten zien op het scherm. Met \texttt{notepad} kan je ini en txt bestanden wijzigen.
\end{description}

De \texttt{Windows} directory bevat heel veel subdirectories, die gaan we niet allemaal behandelen, maar een paar willen we hier wel benoemen:
\begin{description}
\item[\textbackslash Windows\textbackslash System32] Bevat executables en libraries die cruciaal zijn voor de werking van het Windows systeem. Hoewel het lijkt of het 32-bits is, wordt de folder gebruikt voor 64-bits zaken.
\item[\textbackslash Windows\textbackslash SysWOW64] Een 64-bit operating system kan ook 32-bit applicaties draaien. Op een 64-bit Windows versie vind je de \texttt{SysWOW64} folder met daarin de 32-bits files en resources die nodig zijn voor het besturingssysteem. De \texttt{SysWOW64} folder is feitelijk de oude \texttt{System32} folder met daarin de 32-bits zaken.
\end{description}

