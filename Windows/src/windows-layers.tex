Besturingssytemen en software in zijn algemeen, wordt vaak beschreven in lagen, in het Engels layers\index{layer}. Een stapel van deze layers wordt in het Engels dan een stack\index{stack}.

De meest simpele vorm van een besturingssysteem stack is:
\begin{enumerate}
\item user applications\index{user applications}
\item operating system\index{operating system}
\item hardware\index{hardware}
\end{enumerate}

\begin{description}
\item[hardware] zijn de fysieke, tastbare, onderdelen van een computer. Hierbij horen CPU, geheugen, opslagsystemen, etc.
\item[operating system] ook bekent als OS\index{OS}, is een verzameling software die de hardware aanstuurt in opdracht van programma's (applications).
\item[user applications] ook bekent als apps\index{apps}, zijn de stukken software die wij, de gebruikers, gebruiken, zoals Microsoft Office.
\end{description}

De reden van het opdelen in lagen heeft meerdere redenen. Allereerst maakt het een complex geheel dat een operating system is overzichtelijk en makkelijker te begrijpen. Een andere reden is om te laten zien wat er afhankelijk is van elkaar. Bij operating systems heeft het ook te maken met een stukje veiligheid. Het operating system kan en mag tegen de hardware praten, terwijl user applications dat nu juist niet mogen. Het operating system bepaalt wie er wanneer toegang heeft tot een bepaald stuk hardware, waar data in het geheugen zit en waar applicates. Als elke applicatie dat afzonderlijk zou kunnen bepalen zou het een zooitje worden. Het is de taak van het besturingssysteem om te zorgen dat alle taken op een computer op een nette manier worden afgehandeld.

