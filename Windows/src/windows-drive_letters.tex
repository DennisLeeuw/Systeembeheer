Als we in de \texttt{File Explorer} naar \texttt{This PC} gaan dan zien we dat deze machine een Local Disk heeft met tussen haakjes C:. Dit is onze C-drive. Je kan je nu afvragen waar de A en de B-drive gebleven zijn. Dat is geen vreemde vraag. Het Windows besturingssysteem bestaat al lang en in een grijs verleden waren er floppy disks en deze disks moest je plaatsen in een meestal ingebouwde floppy drive. De eerste twee drive letters waren gereserveerd voor deze floppy drives. De eerste harddisk kreeg letter C en dat is altijd zo gebleven.

Dus de C-drive is de disk waarvan het systeem is opgestart. Hierop staat Windows en alles wat daarbij hoort.

