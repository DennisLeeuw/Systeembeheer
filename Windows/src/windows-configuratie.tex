Niet elke gebruiker is hetzelfde en niet elke (werk)omgeving is hetzelfde, daarom is het mogelijk om Windows 11 aan te passen. De plek om te beginnen met aanbrengen van wijzigingen in het systeem is in Settings. Settings (Instellingen) is de opvolger van Control Panel (Configuratiescherm).

Sinds Windows 10 is Microsoft bezig om de configuratie opties vanuit de verschillende Control Panels in Windows te verplaatsen naar Settings of de Group Policy Editor. Deze transitie is nog niet klaar, we beginnen dus in Settings om de nieuwste manier te verkennen. Mochten we het daar niet kunnen vinden dan kijken we naar de Control Panel opties.

Naast deze voor de gebruiker direct toegankelijke (grafische) configuratie lokaties slaan Windows en Applicaties ook veel settings op in de Registry. In dit document gaan we al deze plekken verkennen en onderzoeken om te zien wat we waar kunnen aanpassen.

