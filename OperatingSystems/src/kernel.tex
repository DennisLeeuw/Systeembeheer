De kernel van een besturingssysteem is het deel dat de basis vormt. De kernel zorgt voor de aansturing van memory en CPU, het handeld interrupts af en zorgt voor het doorzetten van data naar de verschillende drivers. De kernel is de regelneef van het besturingssysteem. De kernel biedt interfaces, zogenaamde application programmable interfaces (APIs) aan aan de overige software op het systeem. Het nut hiervan is er een uniforme interface is voor de applicaties en de complexiteit van de hardware wordt afgeschermd. Er zijn bijvoorbeeld honderden verschillende monitoren, het is makkelijk als de applicatie tegen de kernel kan zeggen beeld dit af en deze zorgt dat de data op de juiste manier bij het scherm terecht komt. Het zou vervelend zijn als elke programmeur van elke applicatie elke verkrijgbare monitor zou moeten ondersteunen vanuit zijn applicatie. Daarom zorgt de kernel voor uniforme interfaces.

Een kernel kan monolitisch zijn of modulair. Een monolitische kernel heeft alle drivers die het nodig heeft in de kernel zitten en een modulaire kernel heeft de drivers als modules los op een (boot)disk en deze laadt alleen de modules die het nodig heeft in het geheugen.

Een besturingssysteem bestaat dus uit de kernel en de drivers. Daarnaast zijn er vaak ook nog services of daemons die in het geheugen draaien en die bepaalde taken uitvoeren. Deze services of daemons zijn vergelijkbaar met de programma's op een machine, maar behoren bij het operating system zoals deze uitgeleverd wordt door een leverancier.

