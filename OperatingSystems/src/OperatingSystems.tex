\documentclass[a4paper,12pt,twoside,openright,titlepage]{book}

%Additional packages
\usepackage[utf8]{inputenc}
\usepackage[T1]{fontenc}
\usepackage[dutch,english]{babel}
\usepackage{imakeidx}
\usepackage{syntonly}
\usepackage[official]{eurosym}
%\usepackage[graphicx]
\usepackage{graphicx}
\graphicspath{ {./images/} }
\usepackage{float}
\usepackage{xurl}
\usepackage{hyperref}
\hypersetup{colorlinks=true, linkcolor=blue, citecolor=blue, filecolor=blue, urlcolor=blue, pdftitle=, pdfauthor=, pdfsubject=, pdfkeywords=}
\usepackage{tabularx}
\usepackage{scrextend}
\addtokomafont{labelinglabel}{\sffamily}
\usepackage{listings}
\usepackage{adjustbox}
\usepackage{color}
\usepackage{siunitx} % On Debian do install texlive-science

% Create inch
\DeclareSIUnit[number-unit-product = {}]{\inch}{\textquotedbl}

% Define colors
\definecolor{ashgrey}{rgb}{0.7, 0.75, 0.71}

% Listing style
\lstset{
  backgroundcolor=\color{ashgrey}, % choose the background color; you must add \usepackage{color} or \usepackage{xcolor}; should come as last argument
  basicstyle=\footnotesize,        % the size of the fonts that are used for the code
  breakatwhitespace=true,          % sets if automatic breaks should only happen at whitespace
  breaklines=true,                 % sets automatic line breaking
  extendedchars=true,              % lets you use non-ASCII characters; for 8-bits encodings only, does not work with UTF-8
  frame=single,	                   % adds a frame around the code
  rulecolor=\color{black},         % if not set, the frame-color may be changed on line-breaks within not-black text (e.g. comments (green here))
  keepspaces=true,                 % keeps spaces in text, useful for keeping indentation of code (possibly needs columns=flexible)
  columns=fullflexible,		   % make copy and paste possible
  showstringspaces=false,          % if true show spaces in strings adding particular underscores
  showspaces=false,                % if true show spaces everywhere adding particular underscores; it does not override 'showstringspaces'
}

% Uncomment for production
% \syntaxonly

% Style
\pagestyle{headings}

% Turn on indexing
\makeindex[intoc]

% Define document
\author{D. Leeuw}
\title{Operating Systems}
\date{\today\\v.0.6.0}

\begin{document}
\selectlanguage{dutch}

\maketitle

\copyright\ 2023 Dennis Leeuw\\

\begin{figure}
\includegraphics[width=0.3\textwidth]{CC-BY-SA-NC.png}
\end{figure}

\bigskip

\input{src/CC-licentie}

%%%%%%%%%%%%%%%%%%%
%%% Introductie %%%
%%%%%%%%%%%%%%%%%%%

\frontmatter
\chapter{Over dit Document}
\section{Voorkennis}
Het wordt van de lezer verwacht dat hij de volgende basis kennis heeft:
\begin{itemize}
\item Basis kennis van de werking van een bestandssysteem
\item Basis kennis van de werking van RAID
\end{itemize}

\section{Leerdoelen}
Dit document leert je omgaan met binaire getallen. We behandelen wat binair is en hoe je ermee om gaat. Voor het rekenen met binair beperken we ons tot het optellen, aftrekken en vermenigvuldigen van binaire getallen.



%%%%%%%%%%%%%%%%%
%%% De inhoud %%%
%%%%%%%%%%%%%%%%%
\tableofcontents

% Requires Personal Computing

\mainmatter
\chapter{Inleiding}
Een operating system\index{operating system} (OS\index{OS}) of in het Nederlands besturingssysteem\index{besturingssysteem} is een complex geheel van verschillende stukjes software. Er is niet zoiets als \textbf{het} operating system. We spreken wel van Mac OS X\index{Mac OS X}, Windows\index{Windows} of Linux\index{Linux}, maar eigenlijk zijn dit samenraapsels van verschillende stukken software. De taak van het besturingssysteem is het aansturen van de hardware in opdracht van programma's of gebruikers.

Een gebruiker is een persoon, mens, die via een input-device\index{input-device} de computer een opdracht geeft om iets uit te voeren. Een programma\index{programma} is een stuk software dat geschreven is door een programmeur\index{programmeur} en dat bepaalde opdrachten uitvoert in een bepaalde volgorde. Er zijn een aantal specifieke vormen van programma's:
\begin{description}
	\item[Applicatie]\index{applicatie} Een programma met een grafische interface, die door een gebruiker gebruikt kan worden. Een applicatie werkt op een desktop omgeving.
	\item[Commando]\index{command} Een programma zonder een grafische interface, die door een gebruiker gebruikt kan worden. Een commando werkt op de command line interface (CLI).
	\item[Service]\index{service} Een service is een programma dat op de achtergrond draait en daar bepaalde functies uitvoert zonder dat de gebruiker zich ermee hoeft te bemoeien.
\end{description}


%\section{Een stukje geschiedenis}

% Requires: Kennis van hardware
% Provides: Basis OS
\chapter{Bootstrapping / Booting}
Het is het BIOS/UEFI dat zorg draagt voor de hardware initialisatie en het is uiteindelijk het operating system (besturingssysteem) dat zorgt voor de interface naar de gebruiker en de applicaties. Het proces vanaf het moment dat het BIOS/UEFI klaar is tot het moment dat het operating systeem geladen is heet booting of bootstrapping. Dit bootstrapping proces bestaat uit verschillende stappen:
\begin{enumerate}
\item Het laden van het master-boot record
\item Het starten van de bootloader
\item Het laden en uitvoeren van de kernel
\end{enumerate}

\section{Master Boot Record}
De laatste taak van het BIOS/UEFI is het laden van de Master Boot Record, ofwel de MBR. De MBR is de eerste sector op de bootdisk (512 bytes). De bootdisk kan een USB-stick zijn, een SD-kaart of een SSD of harddisk. Het bootdevice wordt opgegeven in het BIOS en kan daar door de gebruiker gewijzigd worden.

In de masterboot record staat wat de layout is van een disk (partitie-tabel) en welke partitie de kernel van het OS bevat. Het laden van de MBR en daaruit de juiste informatie halen is fase 1 van het boot proces.


\section{Boot Manager}
De boot manager\index{boot manager} is de 2de fase van het boot proces. De meest simpele oplossing is dat de boot manager de kernel van de disk laadt en deze uitvoert. Meestal zit het geheel iets complexer in elkaar. De boot manager kan bijvoorbeeld verschillende opties geven om speciale kernels te laden. Bijvoorbeeld de keuze tussen het opstarten van Linux of Windows. Ook kan de keuze geboden worden om het standaard OS te laden of een rescue omgeving. Er zijn vele varianten mogelijk, maar uiteindelijk is het de taak van de boot manager om een kernel van disk te laden en deze te executeren. Daarna is het aan het operating system.



\chapter{De kernel}
De kernel\index{kernel} van een besturingssysteem is het deel dat de basis vormt. De kernel zorgt voor de aansturing van memory en CPU, het handeld interrupts af en zorgt voor het doorzetten van data naar de verschillende drivers. De kernel is de regelneef van het besturingssysteem. De kernel biedt interfaces\index{interfaces}, zogenaamde application programmable interfaces\index{application programmable interface} (APIs\index{api}) aan aan de overige software op het systeem. Het nut hiervan is er een uniforme interface is voor de applicaties en dat de complexiteit van de hardware wordt afgeschermd. Er zijn bijvoorbeeld honderden verschillende monitoren, het is makkelijk als de applicatie tegen de kernel kan zeggen beeld dit af en dat het OS zorgt dat de data op de juiste manier bij het scherm terecht komt. Het zou vervelend zijn als elke programmeur van elke applicatie elke verkrijgbare monitor zou moeten ondersteunen vanuit zijn applicatie. Daarom zorgt de kernel voor uniforme interfaces.

Een kernel kan monolitisch\index{monolitisch} zijn of modulair\index{modulair}. Een monolitische kernel heeft alle drivers die het nodig heeft in de kernel zitten en een modulaire kernel heeft de drivers als modules los op een (boot)disk en deze laadt alleen de modules die het nodig heeft in het geheugen.

Een besturingssysteem bestaat dus uit de kernel en de drivers. Daarnaast zijn er vaak ook nog services\index{services} of daemons\index{daemons} die in het geheugen draaien en die bepaalde taken uitvoeren. Deze services of daemons zijn vergelijkbaar met gewone programma's, maar deze services behoren bij het operating system zoals deze uitgeleverd wordt door een leverancier.


\section{Drivers}
Drivers zijn specifieke stukjes software die een bepaald deel van de hardware aansturen. Bekende voorbeelden zijn drivers om printers aan te sturen of drivers om het scherm aan te sturen. Deze stukjes software zijn soms heel specifiek zoals een driver voor een HP Color Laser 150 NW of kunnen generiek zijn zoals de driver die printers aanstuurt. Verschillende drivers kunnen dus met elkaar samenwerken. Via de printerdriver kunnen specifieke zaken geregeld worden die bij de specifieke hardware horen, zoals welke papierlade van de printer er gebruikt moet worden.

De print-API neemt de job aan, de printer driver zorgt voor het generieke regelwerk en de HP Color Laser 150 NW driver zorgt ervoor dat data wordt opgemaakt voor deze specifieke printer en stuurt de printer aan.



\chapter{User Interface}
Veelal willen we als mensen computers opdrachten geven. Om dit te kunnen doen moeten we instaat zijn om de computer te besturen, via een muis, toetsenbord of touchscreen. Naast deze input devices moeten we weten hoe we de computer een opdracht geven, door een commando in te typen of door op een bepaald icoontje te klikken. Tot slot moet de computer terug geven wat het resultaat is van de handeling, er moet output zijn naar bijvoorbeeld een scherm of een printer.

Dit alles heeft te maken met de user interface (UI). Het is de interface naar de gebruiker, dit in tegenstelling tot de interface naar de programma's (API). De UI gebruikt de API's om tegen de kernel te praten.


\section{CLI}
De Command Line Interface\index{command line interface} (CLI\index{cli}) is een van de oudste gebruikersinterfaces. De CLI geeft de gebruiker de mogelijkheid om via commando's\index{commando} de computer opdrachten te geven. Om commando's uit te laten voeren heeft de gebruiker en interface nodig. De algemene naam voor deze interface is de command interpreter\index{command interpreter}. De command interpreter is een software laagje tussen de kernel en de gebruiker. De gebruiker typt een commando in, de interpreter voert het uit en de kernel zorgt dat het allemaal geod komt. Elk OS heeft zijn eigen command interpreter(s). Voor Windows hebben we \texttt{cmd}\index{cmd} of \texttt{PowerShell}\index{PowerShell} en voor Linux en Mac OS X hebben we de shell\index{shell}. Er zijn vele verschillende soorten shells, maar de meest gebruikte is \texttt{bash}\index{bash}.

De command line is dus een collectie commando's die door een command interpreter verwerkt wordt. Het vraagt van de gebruiker dat deze de commando's moet kennen en moet weten hoe deze te gebruiken. De leercurve voor het beheersen van deze interface is dan ook vaak stijl aan het begin. Je moet in het begin heel veel leren voordat je iets met de computer kan doen.


\section{GUI}
De meeste desktop en laptop computers zijn voorzien van een operating system met een Graphical User Interface\index{graphical user interface} (GUI\index{gui}). Een GUI wordt aangestuurd met een pointing device\index{pointing device} (o.a. muis\index{muis} of trackpad\index{trackpad}) of een touchscreen\index{touchscreen}. Een grafische interface is makkelijk aan te leren omdat het de mogelijkheid geeft om via plaatjes aan te geven wat een bepaalde functionaliteit doet. Een plaatje met een harddisk slaat informatie op. Een gebruiker hoeft dus geen commando's meer uit zijn hoofd te leren. De ultieme interface zou uit alleen plaatjes bestaan die weergeven wat de functionaliteit is. Dit is echter niet het geval. De meeste interfaces maken nog gebruik van menu's waar functies in zijn ondergebracht, het is dus voor de gebruiker nog noodzakelijk om te weten onder welk menu hij welke functie kan vinden.


\section{Desktop}
Een grafische interface met alleen een pointer heeft niet veel nut. We willen graag een mogelijkheid om applicaties op te starten, een klok om de tijd te zien en een mogelijkeid om bestanden te benaderen. Deze totale omgeving wordt de desktop of het buroblad genoemd.

De desktop is wat de meeste mensen kennen als de GUI. De desktop is meer dan alleen een grafische interface. De desktop is dus een collectie applicaties netzo als de CLI een collectie commando's is.



%%%%%%%%%%%%%%%%%%%%%
%%% Index and End %%%
%%%%%%%%%%%%%%%%%%%%%
\backmatter
\printindex
\end{document}

%%% Last line %%%
