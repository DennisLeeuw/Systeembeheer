Drivers zijn specifieke stukjes software die een bepaald deel van de hardware aansturen. Bekende voorbeelden zijn drivers om printers aan te sturen of drivers om het scherm aan te sturen. Deze stukjes software zijn soms heel specifiek zoals een driver voor een HP Color Laser 150 NW of kunnen generiek zijn zoals de driver die printers aanstuurt. Verschillende drivers kunnen dus met elkaar samenwerken. Via de printerdriver kunnen specifieke zaken geregeld worden die bij de specifieke hardware horen, zoals welke papierlade van de printer er gebruikt moet worden.

De print-API neemt de job aan, de printer driver zorgt voor het generieke regelwerk en de HP Color Laser 150 NW driver zorgt ervoor dat data wordt opgemaakt voor deze specifieke printer en stuurt de printer aan.

