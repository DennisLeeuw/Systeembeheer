De meeste desktop en laptop computers zijn voorzien van een operating system met een Graphical User Interface\index{graphical user interface} (GUI\index{gui}). Een GUI wordt aangestuurd met een pointing device\index{pointing device} (o.a. muis\index{muis} of trackpad\index{trackpad}) of een touchscreen\index{touchscreen}. Een grafische interface is makkelijk aan te leren omdat het de mogelijkheid geeft om via plaatjes aan te geven wat een bepaalde functionaliteit doet. Een plaatje met een harddisk slaat informatie op. Een gebruiker hoeft dus geen commando's meer uit zijn hoofd te leren. De ultieme interface zou uit alleen plaatjes bestaan die weergeven wat de functionaliteit is. Dit is echter niet het geval. De meeste interfaces maken nog gebruik van menu's waar functies in zijn ondergebracht, het is dus voor de gebruiker nog noodzakelijk om te weten onder welk menu hij welke functie kan vinden.

