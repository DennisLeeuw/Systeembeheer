Een computer\index{computer} is een apparaat dat data\index{data} verwerkt. Om data te kunnen verwerken moeten we de computer opdrachten\index{opdrachten} geven en daarvoor gebruiken we tegenwoordig programma's\index{programma}. Een computer kunnen we simpel beschouwen als opgebouwd uit 2 samenhangende onderdelen. We hebben de CPU\index{cpu} (Central Processing Unit\index{central processing unit} of processor\index{processor}) en het RAM\index{RAM} (Random Access Memory \index{Random Access Memory}, geheugen\index{geheugen}). In het geheugen zitten de programma's en de data. De CPU is degene die op basis van opdrachten van een programma data bewerkt uit het geheugen. Het RAM is vluchtig, dat betekent dat als we de computer uitzetten het geheugen leeg wordt. Alle programma's en alle data zijn dan verdwenen uit het geheugen. Willen we onze data (documenten\index{documenten}) bewaren dan zullen we ze moeten opslaan op een apparaat dat niet vluchtig is. We slaan ze op op een opslagapparaat\index{opslagapparaat} (storage device\index{storage device}) zoals een harddisk\index{harddisk}, SSD\index{ssd} of een USB-stick\index{usb-stick}. Ook onze programma's zijn opgeslagen op een storage device en moeten om te kunnen worden gebruikt eerst vanaf het storage device in het geheugen geladen worden voordat we ze hun werk kunnen laten doen.

