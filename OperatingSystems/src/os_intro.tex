Een operating system\index{operating system} (OS\index{OS}) of in het Nederlands besturingssysteem\index{besturingssysteem} is een complex geheel van verschillende stukjes software. Er is niet zoiets als \textbf{het} operating system. We spreken wel van Mac OS X\index{Mac OS X}, Windows\index{Windows} of Linux\index{Linux}, maar eigenlijk zijn dit samenraapsels van verschillende stukken software. De taak van het besturingssysteem is het aansturen van de hardware in opdracht van programma's of gebruikers.

Een gebruiker is een persoon, mens, die via een input-device\index{input-device} de computer een opdracht geeft om iets uit te voeren. Een programma\index{programma} is een stuk software dat geschreven is door een programmeur\index{programmeur} en dat bepaalde opdrachten uitvoert in een bepaalde volgorde. Er zijn een aantal specifieke vormen van programma's:
\begin{description}
	\item[Applicatie]\index{applicatie} Een programma met een grafische interface, die door een gebruiker gebruikt kan worden. Een applicatie werkt op een desktop omgeving.
	\item[Commando]\index{command} Een programma zonder een grafische interface, die door een gebruiker gebruikt kan worden. Een commando werkt op de command line interface (CLI).
	\item[Service]\index{service} Een service is een programma dat op de achtergrond draait en daar bepaalde functies uitvoert zonder dat de gebruiker zich ermee hoeft te bemoeien.
\end{description}

