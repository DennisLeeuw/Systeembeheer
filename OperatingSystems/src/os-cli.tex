De Command Line Interface\index{command line interface} (CLI\index{cli}) is een van de oudste gebruikersinterfaces. De CLI geeft de gebruiker de mogelijkheid om via commando's\index{commando} de computer opdrachten te geven. Om commando's uit te laten voeren heeft de gebruiker en interface nodig. De algemene naam voor deze interface is de command interpreter\index{command interpreter}. De command interpreter is een software laagje tussen de kernel en de gebruiker. De gebruiker typt een commando in, de interpreter voert het uit en de kernel zorgt dat het allemaal geod komt. Elk OS heeft zijn eigen command interpreter(s). Voor Windows hebben we \texttt{cmd}\index{cmd} of \texttt{PowerShell}\index{PowerShell} en voor Linux en Mac OS X hebben we de shell\index{shell}. Er zijn vele verschillende soorten shells, maar de meest gebruikte is \texttt{bash}\index{bash}.

De command line is dus een collectie commando's die door een command interpreter verwerkt wordt. Het vraagt van de gebruiker dat deze de commando's moet kennen en moet weten hoe deze te gebruiken. De leercurve voor het beheersen van deze interface is dan ook vaak stijl aan het begin. Je moet in het begin heel veel leren voordat je iets met de computer kan doen.

