De boot manager\index{boot manager} is de 2de fase van het boot proces. De meest simpele oplossing is dat de boot manager de kernel van de disk laadt en deze uitvoert. Meestal zit het geheel iets complexer in elkaar. De boot manager kan bijvoorbeeld verschillende opties geven om speciale kernels te laden. Bijvoorbeeld de keuze tussen het opstarten van Linux of Windows. Ook kan de keuze geboden worden om het standaard OS te laden of een rescue omgeving. Er zijn vele varianten mogelijk, maar uiteindelijk is het de taak van de boot manager om een kernel van disk te laden en deze te executeren. Daarna is het aan het operating system.

